\documentclass[12pt, a4paper, oneside]{ctexart}
\usepackage{amsmath, amsthm, amssymb, graphicx}
\usepackage[bookmarks=true, colorlinks, citecolor=blue, linkcolor=black]{hyperref}
\usepackage{hyperref} % 引入宏包
\usepackage{caption}
\newtheorem{theorem}{定理}[section]  % [section] 表示按章节编号
\newtheorem{lemma}[theorem]{引理}    % [theorem] 表示与定理共享编号
\newtheorem{proposition}[theorem]{命题}
\newtheorem{corollary}[theorem]{推论}

% 导言区

\title{第3次作业批改情况}
\date{\today}

\begin{document}

\maketitle

\newpage

有界不代表有极限,因此不能直接使用极限的四则运算。
\begin{figure}[htbp]  % htbp 是浮动位置参数
    \centering  % 图片居中
    \includegraphics[width=0.3\textwidth]{1.png}
    \caption{总练习题18}  % 图片标题
    \label{fig:example}  % 标签,用于交叉引用
\end{figure}

错误示范:如果该方法可行,那么$x-sinx$就会等价于0,这是荒谬的,实际上等价于$\frac{1}{6}x^3$当$x\to 0$时。后面学到泰勒展开的时候可以解释什么样的替换是有效的,即用带余项的式子来分析。各位同学需要注意什么时候用等价无穷小才是允许的,胡乱使用就会得到一些错误的分析,有时候你可能瞎撞撞对了,但这不利于你的学习。\\
\begin{figure}[htbp]  % htbp 是浮动位置参数
    \centering  % 图片居中
    \includegraphics[width=0.3\textwidth]{2.png}
    \caption{总练习题23(3)}  % 图片标题
    \label{fig:example}  % 标签,用于交叉引用
\end{figure}

% 不够完整,既然要使用介值定理,就需要先验证题目给的条件已经符合定理使用的前提。\\
% \begin{center}  % htbp 是浮动位置参数
%     \includegraphics[width=0.8\textwidth]{3.png}
%     \captionof{figure}{p73第4题} % 需要 \usepackage{caption}
% \end{center}



我没看出第一行为什么能这样放缩,不是常见的放缩,写清楚怎么来的,否则一律视为错误。第二行注意到$b_{n+1}$的二项式展开多一项(有不少同学都忽略了这一点导致证明错误),就知道即使第三行放缩正确也不能保证不等式方向。\\
\begin{center}  % htbp 是浮动位置参数
    \includegraphics[width=0.7\textwidth]{4.png}
    \captionof{figure}{P74第8题} % 需要 
\end{center}

如果按照下图的式子展开,对比展开项($a_{n+1}$比$a_{n}$多一项,且$a_{n+1}$对应的展开项总是比$a_{n}$大),容易得到第一个命题(单增)的正确性。另一个方向的不等式就不能这样来证了,原因参考上一段,忽略了多出来的一项。我在批改的时候也看到有同学利用均值不等式来证明,一并展示:\\
\begin{figure}[htbp]  % htbp 是浮动位置参数
    \centering  % 图片居中
    \includegraphics[width=0.7\textwidth]{5.png}
    \caption{P74第8题}  % 图片标题
    \label{fig:example}  % 标签,用于交叉引用
\end{figure}

\begin{align*}
    (1+\frac{1}{n})^{n}&=(1+\frac{1}{n})^{n}\times 1 \\
    &\leq [\frac{n\times (1+\frac{1}{n})+1}{n+1}]^{n+1}\\
    \frac{(1+\frac{1}{n+1})^{n+2}}{(1+\frac{1}{n})^{n+1}}&=\frac{(1+\frac{1}{n+1})^{n+1}}{(1+\frac{1}{n})^{n+1}}\times (1+\frac{1}{n+1})\\
    &\leq [\frac{(n+1)\times \frac{(1+\frac{1}{n+1})}{(1+\frac{1}{n})}+1+\frac{1}{n+1}}{n+2}]^{n+2}\\
    &=1
\end{align*}

习题2.1第10题,注意只需要在$x=0$求两侧导数,而且也只有在这点上才可以使用偶函数的对称性。在一点$x$展开的,并且没有指明$x=0$的,一律视为错误。

关于零点存在性定理的版本,根据需要选择:\\
\indent 版本1:设函数 \( f(x) \) 在闭区间 \( [a, b] \) 上连续。如果 \( f(a) \cdot f(b) \leq 0 \),则存在 \( c \in [a, b] \),使得 \( f(c) = 0 \)。\\

版本2:设函数 \( f(x) \) 在闭区间 \( [a, b] \) 上连续。如果 \( f(a) \cdot f(b) < 0 \),则存在 \( c \in (a, b) \),使得 \( f(c) = 0 \)。\\
\indent $Remark$:版本1已经考虑了边界的情况,用的时候就不需要考虑端点是否已经为零点了,可以稍微简化下证明过程。\\

$sin(x)\leq x$成立当且仅当$x\geq 0$(默认即可,要严格证明不是一件简单的事,课本上的证明也不严谨,感兴趣的同学可以搜索相关资料,通过幂级数(实际上就是对应的泰勒级数)定义三角级数,并得到相应的和差化积等公式),不要跟$|sin(x)-sin(y)|\geq |x-y|$混淆,这个不等式是成立的,但是不能直接从$sinx\geq x$得到,这里给出证明:
\begin{align*}
    |sin(x)-sin(y)|&=|2cos(\frac{x+y}{2})sin(\frac{x-y}{2})|\\
    &\leq 2|sin(\frac{x-y}{2})|\\
    &\leq 2 \times |\frac{x-y}{2}|
\end{align*}

这里出一道题:给定一个定义在$(a,b)$上的函数(注意没说是连续的函数),$\forall x \in (a,b),\lim_{y\to x}\frac{f(y)-f(x)}{y-x}>0$(这也意味着这个极限总是存在),证明:$\forall x,y\in (a,b),x<y$,总是有$f(x)< f(y)$.假定你不知道任何关于导数的定理,仅从上述的极限定义出发,尝试解决此问题,感兴趣的同学可以私聊我,可以说说你的想法或者证明,我们可以探讨是否有效。

\end{document}