\documentclass[12pt, a4paper, oneside]{ctexart}
\usepackage{amsmath, amsthm, amssymb, graphicx}
\usepackage[bookmarks=true, colorlinks, citecolor=blue, linkcolor=black]{hyperref}
\usepackage{hyperref} % 引入宏包
\newtheorem{theorem}{定理}[section]  % [section] 表示按章节编号
\newtheorem{lemma}[theorem]{引理}    % [theorem] 表示与定理共享编号
\newtheorem{proposition}[theorem]{命题}
\newtheorem{corollary}[theorem]{推论}

% 导言区

\title{第7次作业批改情况}
\date{\today}

\begin{document}

\maketitle

\newpage
定积分的应用这节公式较多,可以小测或者期末考前临时背诵,特别注意公式里积的变量是什么,不要混淆$dx$和$ds$,也不要忽略积分号前面的常量,比如求极坐标下图形面积,有部分同学没有乘以$\frac{1}{2}$.\\
\indent 习题4.1第2题第2小问是一个很好的补充,罗尔定理,拉格朗日中值定理以及柯西中值定理都是有适用条件的,不要胡乱使用,罗尔定理要求开区间可导(因此不要求端点单侧可导),闭区间连续,这里出一个小练习题:\\
\indent 给定一函数$f(x)$在$[a,b]$连续,$(a,b)$可导,我们知道:
\[
f_{+}'(a)=\lim_{x\to a_+}\frac{f(x)-f(a)}{x-a}
\]
结合拉格朗日中值定理,可以得到:
\[
f_{+}'(a)=\lim_{x\to a_+}f'(\xi)(a<\xi<x)
\]
因此极限存在........(吗)?\\
\indent 下一页给出为什么上述说明是不正确的.\\
\indent 注意以下求导不正确
\[
(x^x)'=x\times x^{x-1}
\]
这个函数相当于$f(x)^{g(x)}$形式,正确求导应该是将其化成$e^{g(x)ln(f(x))}$之后再复合求导。
\newpage
练习题答案:错误的原因是$\lim_{x\to a_+}f'(\xi)(a<\xi<x)$未必存在,第一没法保证导函数就是连续的(这一点可以对比习题4.1的第12题,导函数具有介值性,而连续函数也有介值性,但具有介值性的函数未必连续,可以自行搜索,这种例子应该不太好构造),第二即使是在开区间连续的函数在端点处仍然可能不存在极限,比如$f(x)=\frac{1}{x}$。特别取$f(x)=xsin\frac{1}{x}$就是一个闭区间连续,但在端点处不存在导数的例子。
\end{document}