\documentclass[12pt, a4paper, oneside]{ctexart}
\usepackage{amsmath, amsthm, amssymb, graphicx}
\usepackage[bookmarks=true, colorlinks, citecolor=blue, linkcolor=black]{hyperref}
\usepackage{hyperref} % 引入宏包
\newtheorem{theorem}{定理}[section]  % [section] 表示按章节编号
\newtheorem{lemma}[theorem]{引理}    % [theorem] 表示与定理共享编号
\newtheorem{proposition}[theorem]{命题}
\newtheorem{corollary}[theorem]{推论}

% 导言区

\title{第6次作业批改情况}
\date{\today}

\begin{document}

\maketitle

\newpage
习题3.3 第2题,老师给的答案有误,应为:
\[ x^{3} - \frac{3}{2}x^{2} + 22x - \frac{253}{5}\ln\left|x+3\right| + \frac{53}{5}\ln\left|x-2\right| + C \]

第13题:有的同学采用其它的三角代换,得出的答案虽然跟老师的结果看起来不太一样,但本质上仍是同一个式子。但考试的时候建议采用标准的三角代换,即$t=tan\frac{x}{2}$,否则改卷的人还得额外花功夫验证你的答案,只要双方有一人算错你都可能因此失分。

以下结论很重要,务必熟记:
\[
\; I_n = \int_{0}^{\frac{\pi}{2}} \sin^n x \, dx
= \int_{0}^{\frac{\pi}{2}} \cos^n x \, dx
= 
\begin{cases}
\dfrac{(n-1)!!}{n!!} \cdot \dfrac{\pi}{2}, & n \text{为偶数}, \\[1em]
\dfrac{(n-1)!!}{n!!}, & n \text{为奇数}.
\end{cases}
\]
\indent 一个记忆技巧:该公式的证明基于迭代,每次降两阶,因此,偶数时降到0阶,即:
\[\int_{0}^{\frac{\pi}{2}} \sin^0 x \, dx\]
因此会带一个$\frac{\pi}{2}$,另一个自然带的是1.\\
\indent 最近的几次作业计算题居多,助教没法所有的计算过程都细细算一遍,因此还需要各位同学自行对照答案检查一遍,有不确定的计算过程可以询问助教。


\end{document}