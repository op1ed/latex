\documentclass[12pt, a4paper, oneside]{ctexart}
\usepackage{amsmath, amsthm, amssymb, graphicx}
\usepackage[bookmarks=true, colorlinks, citecolor=blue, linkcolor=black]{hyperref}
\usepackage{hyperref} % 引入宏包
\newtheorem{theorem}{定理}[section]  % [section] 表示按章节编号
\newtheorem{lemma}[theorem]{引理}    % [theorem] 表示与定理共享编号
\newtheorem{proposition}[theorem]{命题}
\newtheorem{corollary}[theorem]{推论}

% 导言区

\title{第2次作业批改情况}
\date{\today}

\begin{document}

\maketitle

\newpage
习题1.4.3(13)解答:
\begin{align*}
% 第1行:写出原始极限,并使用 & 符号设置对齐点
\lim_{x \to +\infty} \frac{a_0 x^m + a_1 x^{m - 1} + \cdots + a_m}{b_0 x^n + b_1 x^{n - 1} + \cdots + b_n}
&= \lim_{x \to +\infty} \frac{\frac{a_0 x^m + a_1 x^{m - 1} + \cdots + a_m}{x^n}}{\frac{b_0 x^n + b_1 x^{n - 1} + \cdots + b_n}{x^n}} \\
% 第2行:分子分母同时除以 x^n
&= \lim_{x \to +\infty} \frac{a_0 x^{m-n} + a_1 x^{m-n-1} + \cdots + \frac{a_m}{x^n}}{b_0 + b_1 x^{-1} + \cdots + \frac{b_n}{x^n}} \\
% 第3行:根据 m 和 n 的关系,分情况讨论得出结论
&=
\begin{cases}
+\infty, & m > n \\
\frac{a_0}{b_0}, & m = n \\
0, & m < n
\end{cases}
\end{align*}
\indent 习题1.4.3(15)提示:利用$(a-b)(a^2+ab+b^2)=a^3-b^3$,即令$a=(1+3x)^{\frac{1}{3}},b=(1-2x)^{\frac{1}{3}}$,剩下的自行补充。\\


利用 $\lim\limits_{x \to 0} \frac{\sin x}{x} = 1$ 及 $\lim\limits_{x \to \infty} \left(1 + \frac{1}{x}\right)^x = e$ 求下列极限:

\begin{align*}
(3) \quad &\lim_{x \to 0} \frac{\tan 3x - \sin 2x}{\sin 5x} \\
&= \lim_{x \to 0} \frac{\tan 3x}{\sin 5x} - \lim_{x \to 0} \frac{\sin 2x}{\sin 5x} \\
&= \frac{3}{5} - \frac{2}{5} \\
&= \frac{1}{5}
\end{align*}

\begin{align*}
(5) \quad &\lim_{x \to a} \frac{\sin x - \sin a}{x - a} \\
&= \lim_{x \to a} \frac{\cos \frac{x + a}{2} \sin \frac{x - a}{2}}{\frac{x - a}{2}} \\
&= \cos a
\end{align*}

\begin{align*}
(6) \quad &\lim_{x \to \infty} \left(1 + \frac{k}{x}\right)^{-x} \\
&= \lim_{x \to \infty} \left(1 + \frac{k}{x}\right)^{\frac{x}{k} \cdot (-k)} \\
&= \left[\lim_{x \to \infty} \left(1 + \frac{k}{x}\right)^{\frac{x}{k}}\right]^{-k} \\
&= e^{-k}
\end{align*}

\begin{align*}
(8) \quad &\lim_{y \to 0} (1 - 5y)^{\frac{1}{y}} \\
&= \left[\lim_{y \to 0} (1 - 5y)^{\frac{1}{-5y}}\right]^{-5} \\
&= e^{-5}
\end{align*}

第5题:$\lim\limits_{x \to a} f(x) = +\infty$:对于任意给定的 $M\in \mathbb{R}$,存在 $\delta > 0$,使得当 $0 < |x - a| < \delta$ 时,$f(x) > M$。

$\lim\limits_{x \to -\infty} f(x) = -\infty$:对于任意给定的 $M$,存在 $\delta$,使得当 $x < \delta$ 时,$f(x) < M$。

习题1.5.4:虽然$f(x)$在$x\in \mathbb{R}-\{0\}$的连续性是显然的,但在解答的过程中还是得说明这一点。另外从这题也可以注意到:连续性是局部的,如果一个函数在某个区间$[a,b]$连续,那么即使延拓该函数到整个实数域上,延拓后的函数一定仍然在$(a,b)$连续,但延拓可能导致端点处的连续性不再成立。


\begin{figure}[htbp]  % htbp 是浮动位置参数
    \centering  % 图片居中
    \includegraphics[width=0.8\textwidth]{1.png}
    \caption{习题1.5.2,错误示范}  % 图片标题
    \label{fig:example}  % 标签,用于交叉引用
\end{figure}
习题1.5.2:证明连续性和使用连续性是不同的事情,证明连续性的时候需要指明对于任意$\epsilon$都存在一个$\delta$,但是使用连续性的时候,需要先指定一个具体的$\epsilon$,然后由连续性随之确定下来$\delta$,本题可以取$\epsilon=f(x_0)/2,f(x_0)/3,f(x_0),...$,不过一般不取$f(x_0)$,大概是遗留下来的习惯,因为一些比较复杂的问题,可能需要多次放缩,多留一点空隙,最后汇总的时候就可以刚好到所要求的精度。当然在本题取这个数没有任何问题。习题1.4.2类似,需要取定一个$\epsilon$再说明.

各位同学如果有任何问题都可以私聊我,在看到后我会尽快回复的。

\end{document}