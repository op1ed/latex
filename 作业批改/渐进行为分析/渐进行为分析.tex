\documentclass[12pt, a4paper, oneside]{ctexart}
\usepackage{amsmath, amsthm, amssymb, graphicx}
\usepackage[bookmarks=true, colorlinks, citecolor=blue, linkcolor=black]{hyperref}
\usepackage{hyperref} % 引入宏包
\usepackage{adjustbox}
\newtheorem{theorem}{定理}[section]  % [section] 表示按章节编号
\newtheorem{lemma}[theorem]{引理}    % [theorem] 表示与定理共享编号
\newtheorem{proposition}[theorem]{命题}
\newtheorem{corollary}[theorem]{推论}

% 导言区

\title{渐进行为分析}
\date{\today}

\begin{document}

\maketitle

\newpage
Model:
\begin{equation}
\left\{
\begin{aligned}
d_S\Delta S-\beta(x)Sf(I)+\gamma(x) I=0 \\
d_I\Delta I+\beta(x)Sf(I)-\gamma(x) I=0
\end{aligned}
\right.
\label{eq1}
\end{equation}
则由$Lou$的最大(最小)值原理,有:
\[\min(\frac{\gamma}{\beta})\frac{1}{f'(0)}\leq S \leq \max(\frac{\gamma}{\beta})\frac{I_{max}}{f(I_{max})}\]
又知道:
\begin{equation}
\adjustbox{max width=\linewidth}{$
\left\{
\begin{alignedat}{2}
-d_I \Delta I &= \beta(x)\Bigl[ \tfrac{N}{|\Omega|}
 + \bigl(\tfrac{d_I}{d_S}-1\bigr)\tfrac{\int_{\Omega} I\,\mathrm{d}x}{|\Omega|}
 - \tfrac{d_I}{d_S} I \Bigr] f(I) - \gamma(x) I, &\quad& x\in\Omega,\\
\frac{\partial I}{\partial \nu} &= 0, && x\in\partial\Omega.
\end{alignedat}
\right.
$}
\end{equation}
同样,利用$Lou$的最大最小值原理,有:
\begin{equation}
    \begin{aligned}
    I\leq \frac{d_S}{d_I}\frac{N}{|\Omega|}+\frac{\int_{\Omega} I dx}{|\Omega|}
    \end{aligned}
    \label{eq3}
\end{equation}



子情况1:$d_S\to 0$\\
\indent 由上述估计,可知$S,I$关于$d_S\to 0$一致有界,由$W^{2,p}$估计及嵌入定理,可以得出$I$的$C^{1,\alpha}$估计关于$d_S$一致有界,结合紧嵌入定理,随着$d_S\to 0$,可以抽出一个子列$d_{S_n}\to 0$,使得对应的解
\[I_n \to I^* \text{ in } C^1(\overline{\Omega})\]
\indent 对式~\eqref{eq1}使用$Harnack$不等式,可知存在不依赖$d_S$的正常数$C$成立以下式子:
\[
C \inf I \geq \sup I
\] 
\indent 因此或者$I^*=0$或者$I^*$是正函数。\\

定义
\begin{equation}
\begin{aligned}
\kappa &= d_S S + d_I I, \\
\overline{S} &= \frac{S}{\kappa}, \\
\overline{I} &= \frac{I}{\kappa}.
\end{aligned}
\end{equation}
\indent 同时积分得到:
\begin{equation}
    \begin{aligned}
    \kappa|\Omega| \leq d_IN
    \end{aligned}
\end{equation}
\indent 即:$\kappa \leq \frac{N}{|\Omega|}d_I$,从中抽出一个子列使得$\kappa_n$收敛到$\kappa^*$,结合$S_n$一致有界,有$I_n$一致收敛到$I^*=\frac{\kappa^*}{d_I}$,即$I^*$为一非负常函数,接下来可分为以下两种情况:\\
(a)存在子序列$d_{S_n}\to 0$,使得对应的$\frac{\kappa}{d_{S_n}}\to \infty$,此时$\overline{I}$一致收敛于$\frac{1}{d_I}$,对式~\eqref{eq1}做变量代换,有以下方程:
\begin{equation}
\begin{aligned}
\frac{d_{S_n}}{\kappa}\Delta S+[\gamma(x)-\beta(x)S\frac{f(I)}{I}]\overline{I}=0
\end{aligned}
\label{eq5}
\end{equation}
结合奇异摄动定理,有$S_n$一致收敛于$S^*=\frac{\gamma I^*}{\beta f(I^*)}$,因此有:
\begin{equation}
    \begin{aligned}
        \int_{\Omega} \frac{\gamma I^*}{\beta f(I^*)}+I^*=N
    \end{aligned}
\end{equation}
注意被积函数关于$I^*$严格递增,因此至多有一解。\\
(b)存在子序列$d_{S_n}\to 0$,使得对应的$\frac{\kappa}{d_{S_n}}\to C^*$,其中$C^*$为正常数(这由下面的式子可以看出,因为$S$有正下界),有:
\begin{equation}
\begin{aligned}
\frac{\kappa}{d_S}=S+d_I\frac{I}{d_S}
\end{aligned}
\label{eq:7}
\end{equation}
claim:$\frac{\|I\|}{d_{S}}$有严格正下界,否则考虑如下方程:
\[
d_I\Delta u+[\beta S \frac{f(d_SI_S)}{d_SI_S}-\gamma]u=0
\]
其中$I_S=\frac{I_n}{d_{S_n}}$,显然$I_S$是上述方程的一个正解,但注意到这同时是一个特征方程,其主特征值满足:
\[
\lambda^*=\min\left\{\int_{\Omega} d_I |\nabla \phi|^2 - (\beta S \frac{f(d_SI_S)}{d_SI_S}-\gamma)\phi^2 dx: \varphi \in H^1(\Omega) \text{ and } \int_{\Omega} \varphi^2 dx = 1 \right\}
\]
由$R_0>1$可知原文(CL)式1.4的主特征值小于0,而由~\eqref{eq:7},可知$S\to \frac{N}{|\Omega|}$,则上述的变分表达式在$d_{S_n}\to 0$时,应该严格小于0,这与0是主特征值矛盾.\\
接下来考虑将式~\eqref{eq1}作$W^{2,p}$估计,因此可以得到$(S,I_S)\to (S^*,I_S^*)$,结合前边的叙述及$Harnack$定理,可知$I_S^*$是正函数,再做$Schauder$估计,结合紧嵌入定理及\eqref{eq5},成立:
\begin{equation}
    \begin{aligned}
        \Delta S^*+[\gamma(x)-\beta(x)S^*f'(0)]I^*=0
    \end{aligned}
\end{equation}

子情况2:$d_S\to \infty$
因为:
\[
S=\frac{N}{|\Omega|}+(\frac{d_I}{d_S}-1)\frac{\int_{\Omega}Idx}{|\Omega|}-\frac{d_I}{d_S}I
\]
因此当$d_S\to \infty$时,$S$的上界一致,然后对~\eqref{eq1}使用$Harnack$不等式,有:
\[
C \inf I\geq \sup I 
\]
因此$I$也有上界。结合椭圆估计及嵌入定理,可以抽出子列使得:
$(S_n,I_n)\to (S^*,I^*)$在$C^{2}(\overline{\Omega})$意义下,断言$I^*=0$不可能成立,证法类似前边$\frac{\|I\|}{d_S}$有严格正下界.其满足以下方程:
\begin{equation}
    \begin{aligned}
    d_I\Delta I^*+\beta(x)\frac{N-\int_{\Omega}I^*}{|\Omega|}f(I^*)-\gamma I^*=0
    \end{aligned}
    \label{eq9}
\end{equation}

这里证明若$\frac{f(x)}{x}$严格递减,则上述方程有且只存在一个正解,存在性由上述说明得到,接证唯一性,考虑带参数的方程:
\[
d_I\Delta u+[\beta(x)\frac{N-\tau}{|\Omega|}\frac{f(u)}{u}-\gamma(x)]u=0
\]
其中$0\leq \tau \leq N$.假设该方程存在正解$u$和$v$,则由于$\epsilon u(0<\epsilon \leq 1)$与$Mu(M\geq 1)$分别是下解与上解(这是由$\frac{f(u)}{u}$的单调性保证的,此处不需要严格单调),因此可以取足够小的$\epsilon$与足够大的$M$使得$v$被夹到这两个函数之间,因此不妨设$u\leq v$,在对应的方程分别乘上$v$与$u$,然后积分,可得:
\[
\int_{\Omega}uv\beta\frac{N-\tau}{|\Omega|}[\frac{f(u)}{u}-\frac{f(v)}{v}]dx=0
\]
由严格单调性即可得到$u=v$,这样便证明了对于固定的$\tau$,该方程至多有一解,另一方面,若有:
\begin{align*}
d_I\Delta u + [\beta \frac{N-\tau_{1}}{|\Omega|}\frac{f(u)}{u}-\gamma]u&=0\\
d_I\Delta v + [\beta \frac{N-\tau_{2}}{|\Omega|}\frac{f(v)}{v}-\gamma]v&=0
\end{align*}
不妨设$\tau_{1}>\tau_2$,将$u$代入到第二个方程,则$u$为下解,取足够大的$M$使得$Mv\geq u$,则由上下解方法及至多唯一性,有$u\leq v\leq Mv$,因此式~\eqref{eq9}唯一性成立。\\
子情况3:$d_I\to \infty,d_S\to \infty$,且$\frac{d_I}{d_S}\to d\in[0,\infty]$\\
如果$d>0$,则由~\eqref{eq3}可知,$I$有上界,则$S$也有上界,则由椭圆估计,可以抽出子列使得$(S_n,I_n)\to (S^*,I^*)$在$C^{2}(\overline{\Omega})$意义下,且$S^*,I^*$均为非负常数,令$\tilde{I}=\frac{I}{\|I\|_\infty}$,则以下方程成立:
\[
d_I\Delta \tilde{I}+[\beta S\frac{f(I)}{I}-\gamma]\tilde{I}=0
\]
利用椭圆估计,不难得到$\tilde{I}\to 1$在$C(\Omega)$意义下,对上述方程积分,取极限得到:
\[
\int_{\Omega}\beta S^* \frac{f(I^*)}{I^*}-\gamma dx=0
\]
即:
\[
S^*=\frac{\int_{\Omega}{\gamma}}{\int_{\Omega}{\beta}}\frac{I^*}{f(I^*)}
\]
结合守恒,便得到:
\[
\frac{\int_{\Omega}{\gamma}}{\int_{\Omega}{\beta}}\frac{I^*}{f(I^*)}+\int_{\Omega}{I^*}=N
\]
严格递增便得到$I^*$的唯一性.
% 即$\frac{I}{d_{S_n}}$一致有界,则有$I^*=0$.claim:$\frac{\|I_n\|_\infty}{d_{S_n}}$有严格正下界,如果$f\in C^{2}$.否则考虑如下方程:
% \begin{equation}
%     \begin{aligned}
%     d_I\Delta I +[\beta f'(0) \frac{N}{|\Omega|}-\gamma-d_I\beta f'(0) I]I=0
%     \end{aligned}
%     \label{eq:8}
% \end{equation}
% 不妨令$I_S=\frac{I_n}{d_{S_n}}$,则将其代入上述方程左侧,有:
% \begin{align*}
% & d_I \Delta I_S + \left[ \beta f'(0) \frac{N}{|\Omega|} - \gamma - d_I \beta f'(0) I_S \right] I_S \\
% = & \beta I_S \frac{f(d_S I_S)}{d_S I_S} \left[ d_II_S - (d_I - d_{S_n}) \frac{\int I_S \, dx}{|\Omega|} - \frac{N}{|\Omega|} \right] \\
% & \quad + \left[ \beta f'(0) \frac{N}{|\Omega|} - d_I \beta f'(0) I_S \right] I_S \\
% = & I_S \left\{ \left[ \frac{f(d_S I_S)}{d_S I_S} - f'(0) \right] \left[ \beta d_I I_S - \beta \frac{N}{|\Omega|} \right] - (d_I - d_{S_n}) \frac{\int I_S \, dx}{|\Omega|} \right\} \\
% \leq & I_S \left\{ \left[ \frac{f(d_S I_S)}{d_S I_S} - f'(0) \right] \left[ \beta d_S I_S - \beta \frac{N}{|\Omega|} \right] - (d_I - d_{S_n}) \frac{C I_S}{|\Omega|} \right\} \\
% \end{align*} 
% 最后一个不等号来自$Harnack$不等式.如果假定$f(x)$是二次连续可微的,则当$n$足够大时可以保证最后一行是小于等于0的,若为$C^{1+\alpha}$的情况,上述放缩没法保证小于等于0.即$\frac{I_n}{d_{S_n}}$是以上方程的上解,又注意到$\epsilon \phi$是一个下解.结合~\eqref{eq:8}有唯一正解,有:
% \[
% \epsilon \phi \leq \hat{I} \leq I_S
% \]
% 积分便得到矛盾。如果$I_S$一致收敛于0,则此时$(S,I)\to (\frac{N}{\Omega},0)$。\\
% 接下来
% % 且对~\eqref{eq1}作$W^{2,p}$估计,结合紧嵌入定理,有$S$在$C^{1}(\overline{\Omega})$收敛到某正连续函数$S^*$,再利用$Schauder$估计及紧嵌入定理,便可得到$(S,I)\to (S^*,I^*)$在$C^{2}(\overline{\Omega})$意义下成立
\end{document}