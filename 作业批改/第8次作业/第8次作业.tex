\documentclass[12pt, a4paper, oneside]{ctexart}
\usepackage{amsmath, amsthm, amssymb, graphicx}
\usepackage[bookmarks=true, colorlinks, citecolor=blue, linkcolor=black]{hyperref}
\usepackage{hyperref} % 引入宏包
\newtheorem{theorem}{定理}[section]  % [section] 表示按章节编号
\newtheorem{lemma}[theorem]{引理}    % [theorem] 表示与定理共享编号
\newtheorem{proposition}[theorem]{命题}
\newtheorem{corollary}[theorem]{推论}

% 导言区

\title{第8次作业批改情况}
\date{\today}

\begin{document}

\maketitle

\newpage
利用泰勒公式求极限的时候,展开的时候一定要保留小$o$项,代表此时是作为等式代入,即:
\[
f(x)=f(a)+...+o(x^k)\\
\]
\[f(x)\sim g(x)\]
是不同的,等价替换本质上也可以视为泰勒展开的应用,只不过直接找到了对应的阶数做替换,而泰勒展开多用于两个同阶的无穷小相减之后来确定对应的无穷小,例如$x\sim sinx$,那么要确定$x-sinx$的等价无穷小呢,此时就可以使用泰勒公式了,$sinx$展开的第一项被$x$消掉后,第二项就作为主导项了。可以说,泰勒展开是整个一元微积分的顶峰,其它方法能求解的极限那么泰勒公式也一定可以,只要你有耐心展开足够多的项.\\
\indent 此外如果一个函数可以泰勒展开到第$k$项,就代表了这个函数在这点有$k$阶导,更高阶的导数则没法保证。反过来,如果一个函数在一点有$k$阶导,那么其也一定能泰勒展开到第$k$项。注意这一段说的泰勒展开都是带小$o$项的展开,如果考虑一个区间内的展开,即给定一个定义在$[a,b]$的函数$f(x)$,足够光滑(即任意阶导数存在),问:
\[
f(x)=f(a)+f'(a)(x-a)+...+f^{(n)}(a)(x-a)^n+...
\]
是否总是成立,这个问题将会在下学期的高数课程中的幂级数一节讨论,因此这块务必好好学习。同时注意区分拉格朗日余项的展开与上述式子的区别。\\
\newpage
以下的证明是不正确的,作为练习,找出证明中错误的部分。(Hint:可能出现\(x_2-x_1\)小的时候,\(f'(c)\)才比较大,此时一大一小相乘是否足够大是未知的)
\begin{figure}[htbp]  % htbp 是浮动位置参数
    \centering  % 图片居中
    \includegraphics[width=0.7\textwidth]{1.png}
    \caption{总练习题8}  % 图片标题
    \label{fig:example}  % 标签,用于交叉引用
\end{figure}
\end{document}