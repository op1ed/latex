\documentclass[12pt, a4paper, oneside]{ctexart}
\usepackage{amsmath, amsthm, amssymb, graphicx}
\usepackage[bookmarks=true, colorlinks, citecolor=blue, linkcolor=black]{hyperref}
\usepackage{hyperref} % 引入宏包
\usepackage{caption}
\newtheorem{theorem}{定理}[section]  % [section] 表示按章节编号
\newtheorem{lemma}[theorem]{引理}    % [theorem] 表示与定理共享编号
\newtheorem{proposition}[theorem]{命题}
\newtheorem{corollary}[theorem]{推论}

% 导言区

\title{第4次作业批改情况}
\author{chen}
\date{\today}

\begin{document}

\maketitle

\newpage

% \begin{center}  % htbp 是浮动位置参数
%     \includegraphics[width=0.8\textwidth]{3.png}
%     \captionof{figure}{p73第4题} % 需要 \usepackage{caption}
% \end{center}

\begin{proposition}
    若$f(x)\geq 0$在$[a,b]$连续,且$\exists x_0\in [a,b],s.t. f(x_0)>0$,则$\int_{a}^{b} f(x)dx>0$.
\end{proposition}
在给出证明之前,给出一个稍显奇怪的例子,定义$Riemann$函数为:
\[
R(x) = 
\begin{cases}
\dfrac{1}{q}, & \text{若 } x = \dfrac{p}{q} \text{ 为最简分数(即 } p, q \in \mathbb{Z}, q > 0, \gcd(p, q) = 1) \\[6pt]
0, & \text{若 } x \text{ 是无理数.}
\end{cases}
\]
\indent 其中$gcd(x,y)$为整数对$(x,y)$的最大公约数.可以验证该函数在$[0,1]$区间上是$Riemann$可积的(要验证可能得花点功夫,感兴趣的同学自行搜索),并且该函数在[0,1]上的积分为$0$,看起来跟上边的命题不太吻合,这是因为不符合处处连续的条件.
\begin{proof}
    首先连续性保证了该函数一定可积,且$\int_{a}^{b} f(x)dx \geq 0$也是平凡的.不妨假设$x_0\in (a,b)$,端点处的情况类似.由连续性,存在区间\[(x_0-\delta,x_0+\delta)\subset [a,b]\]且在这个区间内\[f(x)>\frac{f(x_0)}{2}\]因此有
    \[\int_{a}^{b}f(x)dx\geq \int_{x_0-\delta}^{x_0+\delta} f(x)dx\geq f(x_0)\times \delta>0\]
\end{proof}

习题2.7(18):注意到$(xf(x))'=x^3+1$,如果改用除法得到:
\[
\frac{xf'(x)+f(x)}{x^2}=x+\frac{1}{x^2}
\]
注意左侧的式子不等于:
\[\left( \frac{f(x)}{x} \right)' = \frac{x f'(x) - f(x)}{x^2}\]

习题2.8(5):不少同学的证明都是不完善的,严格的不等号需要用到上边的命题0.1.\\
\indent 习题2.9(2):事实上在证明的过程中仅需要用到$f(x)$在$x_0=a$处右连续,而之所以题目给的条件是$f(x)$在$[a,b]$连续是因为得保证$F_0(x)$有意义,因为不是所有的函数都是可积的,例如$Dirchlet$函数就不是可积的.\\

\indent 习题2.10(2):漏掉该问题后半部分的同学请自行查看老师的解答,这也给我们一个提醒,定理是有适用条件的,不要胡乱使用,否则会得到一些奇怪的结果.再补充一点,被积函数在$x=0$无定义,所以从某种程度上来看对于这个函数的积分定义就不是良好的,可以从两方面看待这件事:从瑕积分的角度来看待这点(这会在下学期学习),另一个方面,忽略掉这一点,或者说重新定义该函数在$x=0$的值,随意取一个数,可以验证该函数仍然可积,并且积分结果与在$x=0$的重新赋值无关,这里举一个简单的例子:
\[
f(x)=
\begin{cases}
    x,x\in (0,1)\\
    \alpha,x=1\\
    \beta,x=0
\end{cases}
\]
\indent $\forall \alpha,\beta \in \mathbb{R}$,f(x)总是可积的,并且积分的值总是$\frac{1}{2}$.\\

\indent 第二章总练习题(3):不能使用求导的四则运算,题目没有说$f(x)$是可导的,一些处处连续但处处不可导的函数:魏尔斯特拉斯函数,Takagi(高木贞治) 函数.这题需要按照导数的定义去求解.

\indent 习题2.5(5),微分符号用$dx$,不要用$\Delta x$.

总练习题(10):这里给出一种比较通用的方法,只要你证明了离散的情况,那么就可以直接从离散形式的不等式推广到积分不等式.
\begin{proof}
    将$[a,b]$$n$等分,$a = x_0 < x_1 < \cdots < x_n = b$,则:
    \begin{align*}
        \left( \int_a^b f(x)g(x)\,dx \right)^2 
        & = \lim_{n \to \infty} 
        \left[ \sum_{i=1}^n f(x_i) g(x_i) \frac{b-a}{n} \right]^2\\
        & \leq 
        \lim_{n \to \infty}
        \sum_{i=1}^n f^2(x_i)\frac{b-a}{n}
        \cdot
        \sum_{i=1}^n g^2(x_i)\frac{b-a}{n}\\
        & = \left( \int_a^b f^2(x)\,dx \right)
        \left( \int_a^b g^2(x)\,dx \right)\\
    \end{align*}
    第一个不等号由 Cauchy 不等式得到,只要令:\[a_i = f(x_i)\sqrt{\tfrac{b-a}{n}}
    ,b_i = g(x_i)\sqrt{\tfrac{b-a}{n}}\]
    注意不要漏掉极限符号,毕竟积分就是通过黎曼和来定义的,上边的黎曼函数积分也只能通过这种定义来计算,无法直接使用牛顿-莱布尼兹公式.值得一提的是,Cauchy不等式也可以用类似的判别法来证明,即老师的解答里边对于积分形式的证明。
\end{proof}

习题2.7(18),有同学指出最后的答案代入$x=0$可以得到$c=0$,,如果题目有说$f(x)$是定义在整个实数域上的函数,那么没有问题,如果仅定义在$x>0$的区间,那么代入$x=0$就不行了。本题确实有歧义,一般来说都是默认整个实数域.

习题2.5(10)(3):有同学直接把$x$和$y$给解出来,这当然也是可行的,但解的时候错误使用了以下公式:
\[
arccos(cos(\theta))=\theta
\]
这个式子并不成立,因为$arccos(x)$的值域为$[0,\pi]$,只有$\theta \in [0,\pi]$时才成立。

\end{document}